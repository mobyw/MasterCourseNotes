\documentclass[cn, hazy, blue, normal, 12pt]{elegantnote}

\title{高级计算机系统结构例题讲解}
\author{Mobyw}
\version{1.0}
\date{\zhtoday}

\usepackage{listings}
\usepackage{tikz}
\usepackage{hyperref}
\usepackage{pgfplots}
\usepackage{bookmark}
\usepackage{multirow}
\usepackage{tabularx}
\usepackage[verbose]{xsim}
\usepackage[enableskew,vcentermath]{youngtab}
\usepackage[
    type={CC},
    modifier={by-nc-sa},
    version={4.0},
]{doclicense}

\usetikzlibrary{patterns}
\pgfplotsset{compat=1.18}

\tikzset{
    box/.style ={
            rectangle,              % 矩形节点
            rounded corners = 5pt,  % 圆角
            minimum width   = 50pt, % 最小宽度
            minimum height  = 20pt, % 最小高度
            inner sep = 5pt,        % 文字和边框的距离
            draw=blue               % 边框颜色
        }
}

\begin{document}

\maketitle

% \setlength{\lineskip}{1.5em}
% \setlength{\parskip}{0pt}

\doclicenseThis

本文档为高级计算机系统结构各章节的例题,由于部分答案为个人编撰,难免会出现错误,请保证使用 \href{https://github.com/mobyw/MasterCourseNotes/blob/master/AdvancedComputerSystemArchitecture/AdvancedComputerSystemArchitectureExamples.pdf}{GitHub仓库} 所发布的最新版本. 如遇问题可在 GitHub 上发布 Issue.


\section{量化设计与分析基础}

\begin{exercise}

    设一台计算机运行某程序的CPU时间如下表:

    \begin{table}[htbp]
        \centering
        \notag
        \begin{tabular}{ccccc}
            \hline
            浮点指令 & 整数指令  & 读/写指令 & 分支指令 & 总时间   \\
            \hline
            60 s & 100 s & 40 s  & 40 s & 240 s \\
            \hline
        \end{tabular}
    \end{table}

    试计算:

    (1)如浮点指令执行时间减少50\%,总时间减少百分之多少?总时间减少后的加速比为多少?

    (2)如总时间减少15\%,只减少整数指令时间,整数指令时间减少百分之多少?

    (3)如只减少分支指令时间,总时间能否减少20\%?

\end{exercise}

\begin{solution}[print=true]



\end{solution}

\begin{exercise}

    假如你的公司需要选购 Opteron 或 Itanium2。公司的应用情况是:50\% 的时间运行类似于 mesa 的应用程序,25\% 的时间运行类似于 applu 应用程序,25\% 的时间运行类似于 lucas 的应用程序。 后面的表提供了 Opteron 和 Itanium 的运行测试程序的信息。

    (1)如果仅根据SPEC总体性能进行选择,你选择哪一种微处理器?为什么?

    (2)计算公司混合应用程序的 Itanium/Opteron SPECRatio 加权平均值是多少?按照这个结果应该选择哪一种微处理器?

\end{exercise}

\begin{solution}[print=true]



\end{solution}


\section{指令系统原理与示例}



\section{单周期 MIPS 处理器的设计}

\begin{exercise}

    MIPS单周期CPU数据通路及控制信号如下图所示。

    (1)试分析下列指令在寄存器 R6 与 R7 内容相等时,下表中针对每条指令的各控制信号的取值。若某控制信号与某指令无关则用 X 表示。

    \lstinline{AND  R1, R2, R3}\par
    \lstinline{SW   R4, 10(R5)}\par
    \lstinline{BEQ  R6, R7, OK}

    (2)假设指令存储器延迟是 300 ps,加法器延迟是 80 ps,多路选择器延迟是 20 ps,ALU 延迟是 100 ps,寄存器堆延迟 200 ps,数据存储器延迟 300 ps,忽略符号扩展器的延迟。试计算以下指令的关键路径延迟是多少?

    \lstinline{LW   R2, 20(R1)}

\end{exercise}




\begin{solution}[print=true]



\end{solution}



\section{流水线技术及指令级并行}



\section{存储系统}



\end{document}
