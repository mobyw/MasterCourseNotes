\documentclass[cn, hazy, blue, normal, 12pt]{elegantnote}

\title{形式化方法笔记}
\author{Mobyw}
\institute{Created by Elegant\LaTeX{}}
\version{1.0}
\date{\zhtoday}

\usepackage{tikz}
\usepackage{pgfplots}
\usepackage{bookmark}
\usepackage{multirow}
\usepackage{tabularx}
\usepackage{graphicx}

\pgfplotsset{compat=1.18}

\begin{document}

\maketitle

\section{基础}

\subsection{自然语言和命题逻辑语句的转换}



\subsection{自然语言和谓词逻辑语句的转换}



\subsection{语法分析树的绘制}

符号优先级:

\begin{enumerate}
    \item 括号:$()$.
    \item 一元连接词:$\neg$、$\exists$、$\forall$、($\mathrm{X}$、$\mathrm{F}$、$\mathrm{G}$)或($\mathrm{AX}$、$\mathrm{AF}$、$\mathrm{AG}$、$\mathrm{EX}$、$\mathrm{EF}$、$\mathrm{EG}$).
    \item 二元连接词:$\wedge$、$\vee$.
    \item 二元连接词:$\rightarrow$、$\mathrm{U}$或($\mathrm{AU}$、$\mathrm{EU}$).
\end{enumerate}

\section{命题逻辑}

\subsection{矢列的有效性证明}



\subsection{根据真值表构造 CNF 形式的公式}

要根据真值表构造CNF(合取范式)形式的公式,需要遵循以下步骤:

\begin{enumerate}
    \item 确定真值表中输出为假的行的集合.
    \item 对于每个真值表中输出为假的行,将该行的输入变量的取反作为一个析取子句.
    \item 对所有析取子句进行合取运算,得到 CNF 形式的公式.
\end{enumerate}

假设有以下真值表:

\begin{center}
    \begin{tabular}{ccc|c}
        \hline
        A & B & C & F \\
        \hline
        F & F & F & F \\
        F & F & T & T \\
        F & T & F & F \\
        F & T & T & T \\
        T & F & F & T \\
        T & F & T & F \\
        T & T & F & F \\
        T & T & T & T \\
        \hline
    \end{tabular}
\end{center}

\begin{enumerate}
    \item 确定真值表中输出为假的行的集合:$\{ 1, 3, 6, 7 \}$.
    \item 对于每个真值表中输出为假的行,将该行的输入变量的取反值作为一个析取子句:
          \begin{enumerate}
              \item 第1行:$(A \lor B \lor C)$
              \item 第3行:$(A \lor \neg B \lor C)$
              \item 第6行:$(\neg A \lor B \lor \neg C)$
              \item 第7行:$(\neg A \lor \neg B \lor C)$
          \end{enumerate}
    \item 对所有析取子句进行合取运算,得到 CNF 形式的公式:\\
          $(A \lor B \lor C) \land (A \lor \neg B \lor C) \land (\neg A \lor B \lor \neg C) \land (\neg A \lor \neg B \lor C)$
\end{enumerate}

\section{谓词逻辑}

\subsection{矢列的有效性证明}



\subsection{模型有效性判断}



\section{时态逻辑}

\subsection{迁移系统与计算路径}



\subsection{路径与 LTL/CTL 的满足关系}



\section{模型检测}

\subsection{标记算法}



\section{程序验证}

\subsection{霍尔三元组验证}



\subsection{if 语句验证}



\subsection{while 语句部分正确性验证}



\subsection{完全准确性验证}



\end{document}
