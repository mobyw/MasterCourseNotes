\documentclass[cn, hazy, blue, normal, 12pt]{elegantnote}

\title{组合数学例题讲解}
\author{Mobyw}
\version{1.0}
\date{\zhtoday}

\usepackage{tikz}
\usepackage{pgfplots}
\usepackage{bookmark}
\usepackage{multirow}
\usepackage{tabularx}
\usepackage[verbose]{xsim}

\pgfplotsset{compat=1.18}

\tikzset{
    box/.style ={
        rectangle,              % 矩形节点
        rounded corners = 5pt,  % 圆角
        minimum width   = 50pt, % 最小宽度
        minimum height  = 20pt, % 最小高度
        inner sep = 5pt,        % 文字和边框的距离
        draw=blue               % 边框颜色
    }
}

\begin{document}

\maketitle

% \setlength{\lineskip}{1.5em}
% \setlength{\parskip}{0pt}


本文为组合数学各章节的例题,由于部分答案为个人编撰,难免会出现错误,请保证使用 \href{https://github.com/mobyw/MasterCourseNotes/CombinatorialMathematics}{GitHub仓库} 所发布的最新版本. 如遇问题可在 GitHub 上发布 Issue.


\section{排列、组合及二项式定理}

\begin{exercise}

    奔赴抗疫,全国 4 个片区共有 68个 医疗队,其中西南片区有 10 个,中部片区有 18 个,北方片区有 18个,东部片区有22 个. 假定同一片区的各 个医疗队不加以区别,现在要从中选取 27个医疗队入围. 考虑到不同片区 的特殊情况,要求西南片区至少入围 4 个医疗队,北方片区至少入围 7 个医疗队,其他片区至少各入围 2 个医疗队,问理论上有多少种不同的选取方案?

\end{exercise}

\begin{solution}[print=true]

    Solution.

\end{solution}


\section{容斥原理}

\begin{exercise}

    求方程:

    \begin{equation}
        \notag
        \left\{\begin{array}{l}
            x_{1}+x_{2}+4 x_{3}+x_{4}=160 \\
            2 \leq x_{3} \leq 10, x_{4} \leq 3
        \end{array}\right.
    \end{equation}

    正整数解的个数.

\end{exercise}

\begin{solution}[print=true]

    Solution.

\end{solution}

\begin{exercise}

    求方程:

    \begin{equation}
        \notag
        \left\{\begin{array}{l}
            x_{1}+3 x_{2}+x_{3}+x_{4}=160 \\
            3 \leq x_{2} \leq 10, x_{3} \leq 3
        \end{array}\right.
    \end{equation}

    正整数解的个数.

\end{exercise}

\begin{solution}[print=true]

    Solution.

\end{solution}


\section{鸽笼原理与 Ramsey 定理}

\begin{exercise}

    证明 $11$ 个人中必定有 $4$ 个人彼此相认或 $3$ 个人彼此不相识.

\end{exercise}

\begin{solution}[print=true]

    Solution.

\end{solution}

\begin{exercise}

    证明 $R(3, 3) < 7$.

\end{exercise}

\begin{solution}[print=true]

    Solution.

\end{solution}

\begin{exercise}

    证明 $S_{2}(n, n-1) = \dfrac{n(n-1)}{2}$.

\end{exercise}

\begin{solution}[print=true]

    Solution.

\end{solution}


\section{母函数}

\begin{exercise}

    正偶数 $k_1, k_2, ..., k_n$ 满足 $k_i \neq k_j, i \neq j$. 写出求将正整数 $r$ 分解为 $k_1, k_2, ..., k_n$ 的和的方法数的算法,要求 $k_i$ 最多可被选中三次.

\end{exercise}

\begin{solution}[print=true]

    Solution.

\end{solution}

\begin{exercise}

    求不包含 $3, 5, 7$,出现偶数次 $1, 2$,至少出现两次 $4, 8$ 的 $r$ 位十进制数的个数.

\end{exercise}

\begin{solution}[print=true]

    Solution.

\end{solution}


\section{递归关系}

\begin{exercise}

    求解递归关系:

    \begin{equation}
        \notag
        \left\{\begin{array}{l}
            a_{n}-2 a_{n-1}-3 a_{n-2}=2 \cdot 3^{n} \\
            a_{0}=1, a_{1}=2
        \end{array}\right.
    \end{equation}

\end{exercise}

\begin{solution}[print=true]

    Solution.

\end{solution}

\begin{exercise}

    求解递归关系:

    \begin{equation}
        \notag
        \left\{\begin{array}{l}
            a_{n}=3 a_{n-1}+4 a_{n-2}+2 \cdot 4^{n} \\
            a_{0}=1, a_{1}=1
        \end{array}\right.
    \end{equation}

\end{exercise}

\begin{solution}[print=true]

    Solution.

\end{solution}


\end{document}
